\pdfoutput=1
 
\documentclass{l4proj}
\usepackage[usenames,dvipsnames]{xcolor}
\usepackage{enumitem}
\usepackage{hyperref}
\usepackage{titlesec}
\titlespacing{\subsubsection}{0pt}{6pt}{6pt}
\definecolor{properBlue}{HTML}{336699}
\hypersetup{
    colorlinks,
    citecolor=properBlue,
    filecolor=properBlue,
    linkcolor=properBlue,
    urlcolor=properBlue
}
 
\begin{document}
\title{Package Recommendation Engine}
\author{Keir Alexander Smith}
\date{March 15, 2015}
\maketitle
 
\begin{abstract}
I did things.
\end{abstract}
 
%\educationalconsent
\tableofcontents
 
%%%%%%%%%%%%%%%%%%
%                %
%  INTRODUCTION  %
%                %
%%%%%%%%%%%%%%%%%%
 
\chapter{Introduction}
\pagenumbering{arabic}
 
\section{Problem overview}
In a modern operating system there exists many packages for end users to install and this collection grows every day. Finding useful packages to install can be a labourous task, often involving the use of the internet to track down the package the user has been looking for, if it exists.\\
Furthermore there exists little support for installing packages commonly installed side by side. For example a user who has vim installed also installs JDK, the user may not be aware of the existence of a java plugin for vim which is extreamly useful.
 
\section{Aims}
This project aims to attmept to address the issues discussed above. Foremostly the problem of finding new packages by offering a powerful recommendation system for users to discover packages. 
 
\section{Motivation}
Package management is an interesting and very useful tool for many users, however the basic implementation has been static for many years. With the addidition of this tool, we could see a decrease in users having to use search engines to look for packages they should be able to discover easily on commandline.

\section{Report outline}
This report will begin by looking at the research undertaken at the outset of the project, then continue into design of the system. This will lead into the implementation and finally the evaluation and conclusion.
 
%%%%%%%%%%%%%%%%
%              %
%  BACKGROUND  %
%              %
%%%%%%%%%%%%%%%%
 
\chapter{Background}
 
%%%%%%%%%%%%
%          %
%  DESIGN  %
%          %
%%%%%%%%%%%%
 
\chapter{Design}
 
%%%%%%%%%%%%%%%%%%%%
%                  %
%  IMPLEMENTATION  %
%                  %
%%%%%%%%%%%%%%%%%%%%
 
\chapter{Implementation}
 
%%%%%%%%%%%%%%%%
%              %
%  EVALUATION  %
%              %
%%%%%%%%%%%%%%%%
 
\chapter{Evaluation}
 
%%%%%%%%%%%%%%%%
%              %
%  CONCLUSION  %
%              %
%%%%%%%%%%%%%%%%
 
\chapter{Conclusion}
 
%%%%%%%%%%%%%%%%
%              %
%  APPENDICES  %
%              %
%%%%%%%%%%%%%%%%
 
\begin{appendices}
 
\chapter{Name of the first appendix}
 
\end{appendices}
 
%%%%%%%%%%%%%%%%%%%
%                 %
%  BIBLIOGRAPHY   %
%                 %
%%%%%%%%%%%%%%%%%%%
 
\bibliographystyle{plain}
\bibliography{bib}
\end{document}